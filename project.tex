\documentclass[letterpaper,12pt]{article}

%math and margin packages
\usepackage{amsmath,amsfonts,amssymb,amsthm}
\DeclareMathOperator{\sech}{sech}
\usepackage{braket}
\usepackage[margin=1.0in]{geometry}
\usepackage{bbold}
\usepackage{braket}
\usepackage{ragged2e}
\usepackage{tikz}
\usetikzlibrary{angles,quotes}
\usepackage{tkz-euclide}
\usepackage{svg}
\usepackage{setspace}
\allowdisplaybreaks
\doublespacing

\usepackage{titling}
\renewcommand\maketitlehooka{\null\mbox{}\vfill}
\renewcommand\maketitlehookd{\vfill\null}

\title{
\normalfont \normalsize 
\textsc{APPM 2360 - Intro Diff Eq W/Lin Alg \hfill Fall 2024} \\
[10pt] 
\rule{\linewidth}{0.5pt} \\[6pt] 
\huge Project 1 - Fish Population Modeling \\
\rule{\linewidth}{2pt}  \\[10pt]
}
\date{October 08, 2024}
\author{Daniel Alemayehu, Eli Grundberg}

\begin{document}
\begin{titlingpage}
\maketitle
\end{titlingpage}

\newpage

\section*{Introduction}
We interested in stocking a local neighborhood pond with a species of trout such that people can enjoy finishing in said pond. However, we want to prevent the pond from being overfished. More specifically, we want to make sure there is always fish in the pond such that anyone who goes fishing has a nonzero chance of catching a fish (because there are no fish). To accomplish this, we'll analyze different species of trouts to see which population is best suitable for recreational fishing in this neighborhood.
\section*{1}
\section*{2}
\section*{3} 
\section*{Conclusion}
We have found that the best species to stock the lake is \underline{\hspace{3cm}} because \underline{\hspace{3cm}}. However, our model doesn't account for a variety of factors including, \underline{\hspace{3cm}}. blah blah blah
\section*{References}
\end{document}
